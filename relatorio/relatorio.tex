\documentclass[12pt, a4paper]{article}
\usepackage[utf8]{inputenc}
\usepackage[brazilian]{babel} % Hifenização e dicionário
\usepackage[left=3.00cm, right=2.00cm, top=3.00cm, bottom=2.00cm]{geometry}
\usepackage{enumitem} % Para itemsep etc
\usepackage{longtable} % Dependência do longtabu
\usepackage{tabu} % Para melhor criação de tabelas
\usepackage{zi4} % Para fonte de códigos
\usepackage{listings} % Para códigos
\usepackage{lstautogobble} % Códigos indentados corretamente
\usepackage{color} % Para coloração de códigos
\usepackage{mathpazo} % Palatino
\usepackage{parskip} % Linha em branco entre parágrafos em vez de recuo
\usepackage{graphicx}
\usepackage{verbatim} % Para comentários
\usepackage{booktabs}
\usepackage{amsmath} % bmod
\usepackage[breaklinks]{hyperref}

\DeclareGraphicsExtensions{.pdf,.png}

\newcommand{\code}[1]{{\lstinline{#1}}}

\usepackage{listings}
\lstset{
    autogobble,
    columns=fullflexible,
    showspaces=false,
    showtabs=false,
    breaklines=true,
    showstringspaces=false,
    breakatwhitespace=true,
    escapeinside={(*@}{@*)},
    basicstyle=\ttfamily\footnotesize,
    frame=l,
    framesep=12pt,
    xleftmargin=12pt,
    tabsize=4,
    captionpos=b
}

\begin{document}
\begin{center}
    \textsc{Universidade Federal do Rio Grande do Norte} \\
    \textsc{Departamento de Informática e Matemática Aplicada}
\end{center}

\bigskip

\begin{tabular}{@{}ll@{}}
    \emph{Disciplina:} & DIM0612 --- Programação Concorrente \\
    \emph{Docente:}    & Everton Ranielly de Sousa Cavalcante \\
    \emph{Discente:}   & Felipe Cortez de Sá \\
\end{tabular}

\bigskip

\begin{center}
\large Sincronização de um banheiro unissex virtual
\end{center}

\section{Introdução}
Este relatório descreve a implementação de algoritmos para sincronização de
\emph{threads} usando duas estratégias diferentes.

\section{Modelo}
As partes comuns às duas estratégias de sincronização estão no arquivo
\code{common.py}.

O banheiro é implementando pela classe \code{Toilet}, que possui internamente
um contador \code{counter} responsável por guardar o número de pessoas usando o
toalete e uma lista de tamanho fixo $ l $, sua capacidade máxima. O método
\code{enter} recebe como parâmetro um objeto da classe \code{Person} e troca o
primeiro espaço vago da lista pelo objeto. O método \code{leave} recebe como
parâmetro uma pessoa a ser retirada do banheiro, atualizando a lista com um
espaço vago onde a pessoa estava. Os espaços vagos são codificados como uma
\emph{string} \code{EMPTYSTR}.

Uma pessoa é representada como instância da classe \code{Person}, que possui um
atributo \code{gender}, podendo assumir os valores de \emph{string} \code{'M'}
ou \code{'F'}, determinados aleatoriamente na instanciação. Além disso, cada
pessoa recebe um nome aleatório para visualização durante a execução do
programa.

A função \code{main} recebe como parâmetro uma classe que implementa o método
\code{personThread}, responsável pela lógica de sincronização. Os arquivos
\code{locks.py} e \code{semaphores.py} implementam as classes e chamam a função
\code{main}. Assim, os programas funcionam independentemente, cada um com seu
próprio ponto de entrada \code{main}. 

Foi utilizada a biblioteca \code{argparse} para leitura de parâmetros de linha
de comando. O único parâmetro lido é \code{l}, a capacidade do banheiro. Os
programas podem ser executados através dos comandos \code{python3
[semaphores|locks].py -l 5}.

\section{\emph{Lock}}
O \emph{lock} é adquirido por uma thread sempre ao ler o estado do banheiro. Se
o banheiro estiver vazio ou tiver apenas pessoas do mesmo sexo e não estiver
lotado, a pessoa entra no banheiro e libera o \emph{lock}. Caso contrário, a
pessoa não entra e libera o \emph{lock}, repetindo as tentativas. Em outras
palavras, o \emph{lock} apenas serve como uma maneira de limitar o acesso ao
objeto \code{toilet}, transformando a leitura e escrita numa operação atômica.

\section{Semáforo e variável de condição}
Foi utilizada a classe \code{BoundedSemaphore} do Python, inicializada com
limite determinado pelo usuário representando o número de pessoas que podem
utilizar o banheiro simultaneamente. Para não permitir que pessoas do mesmo
sexo entrem no banheiro, foi utilizada uma variável de condição \code{empty},
que faz a \emph{thread} esperar se o banheiro estiver ocupado por pessoas do
sexo oposto. Ao final da execução de uma \emph{thread}, as \emph{threads} em
espera são notificadas através do método \code{notify_all} caso o banheiro
fique vazio.

\end{document}
